%---------------------------------------------------------------------------%
%->> 中英文摘要
%---------------------------------------------------------------------------%

% 中文摘要页
\begin{center}
    \vspace*{1cm}  % 上空一行
    {\zihao{-3}\heiti\bfseries 二甲醚清洁燃料均质压燃燃烧数值模拟研究}
    
    \vspace{1cm}  % 下空一行
\end{center}

\begin{center}
    {\zihao{4}\heiti\bfseries 摘\quad 要}
\end{center}

\vspace{0.5cm}  % 摘要标题与正文之间的间距

% 设置摘要正文格式:五号宋体,首行缩进2字符,单倍行距
\setlength{\parindent}{2em}  % 首行缩进2字符
{\zihao{5}\songti\linespread{1.0}\selectfont
均质充量压缩着火(HCCI)燃烧,作为一种能有效实现高效低污染的燃烧方式,能够使发动机同时保持较高的燃油经济性和动力性能,而且能有效降低发动机的NOx和碳烟排放。此外HCCI燃烧的一个显著特点是燃料的着火时刻和燃烧过程主要受化学动力学控制,基于这个特点,发动机结构参数和工况的改变将显著地影响着HCCI发动机的着火和燃烧过程。本文以新型发动机代用燃料二甲醚(DME)为例,对HCCI发动机燃用DME的着火和燃烧过程进行了研究。研究采用由美国Lawrence Livermore国家实验室提出的DME详细化学动力学反应机理及其开发的HCT化学动力学程序,且DME的详细氧化机理包括399个基元反应,涉及79个组分。为考虑壁面传热的影响,在HCT程序中增加了壁面传热子模型。采用该方法研究了压缩比、燃空当量比、进气充量加热、发动机转速、EGR和燃料添加剂等因素对HCCI着火和燃烧的影响。结果表明,DME的HCCI燃烧过程有明显的低温反应放热和高温反应放热两阶段;增大压缩比、燃空当量比、提高进气充量温度、添加H2O2、H2、CO使着火提前;提高发动机转速、采用冷却EGR、添加CH4、CH3OH使着火滞后。
}

\vspace{0.5cm}  % 正文与关键词之间的间距

\noindent
{\zihao{-4}\heiti\bfseries 关键词:}\hspace{0.5em}
{\zihao{5}\songti
均质充量压缩着火,化学动力学,数值模拟,二甲醚,EGR
}

\vspace{1cm}  % 中文摘要与英文摘要之间的间距

% 英文摘要页
\newpage
\begin{center}
    \vspace*{1cm}  % 上空一行
    {\zihao{3}\bfseries\rmfamily\MakeUppercase{Numerical Simulation of Homogeneous Charge Compression Ignition Combustion Using Dimethyl Ether as Clean Fuel}}
  
    \vspace{1cm}  % 下空一行
\end{center}

\begin{center}
    {\zihao{4}\bfseries\rmfamily ABSTRACT}
\end{center}

\vspace{0.5cm}  % ABSTRACT标题与正文之间的间距

% 设置英文摘要正文格式:五号字体,首行缩进2字符,单倍行距
\setlength{\parindent}{2em}  % 首行缩进2字符
{\zihao{5}\rmfamily\linespread{1.0}\selectfont
Homogeneous Charge Compression Ignition (HCCI) combustion, as an effective combustion method to achieve high efficiency and low pollution, can enable engines to maintain high fuel economy and power performance while effectively reducing NOx and soot emissions. A significant characteristic of HCCI combustion is that the ignition timing and combustion process are primarily controlled by chemical kinetics. Based on this characteristic, changes in engine structural parameters and operating conditions significantly affect the ignition and combustion processes of HCCI engines. This paper takes the new engine alternative fuel dimethyl ether (DME) as an example to study the ignition and combustion processes of HCCI engines using DME. The research employs the detailed chemical kinetic reaction mechanism of DME proposed by the Lawrence Livermore National Laboratory in the United States and the HCT chemical kinetics program developed by them. The detailed oxidation mechanism of DME includes 399 elementary reactions involving 79 species. To account for the influence of wall heat transfer, a wall heat transfer sub-model was added to the HCT program. This method was used to study the effects of compression ratio, fuel-air equivalence ratio, intake charge heating, engine speed, EGR, and fuel additives on HCCI ignition and combustion. The results show that the HCCI combustion process of DME has two distinct stages: low-temperature reaction heat release and high-temperature reaction heat release. Increasing the compression ratio, fuel-air equivalence ratio, intake charge temperature, and adding H2O2, H2, and CO advance the ignition timing; while increasing engine speed, using cooled EGR, and adding CH4 and CH3OH retard the ignition timing.
}

\vspace{0.5cm}  % 正文与关键词之间的间距

\noindent
{\zihao{-4}\bfseries\rmfamily Key words:}\hspace{0.5em}
{\zihao{5}\rmfamily
Homogeneous Charge Compression Ignition, chemical kinetics, numerical simulation, dimethyl ether, EGR
}
%---------------------------------------------------------------------------%