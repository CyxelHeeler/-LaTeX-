% LaTeX宏包配置文件 - 上海科技大学本科毕业论文模板

% 忽略警告
\usepackage{xpatch}
\ExplSyntaxOn
    \xpatchcmd\__xeCJK_check_family:n{\__xeCJK_warning:nxx}{\__xeCJK_info:nxx}{}{}
\ExplSyntaxOff

% 页面布局设置
\usepackage{geometry}  % 页面边距和布局控制
\geometry{
    a4paper,           % A4纸张尺寸
    left=2.5cm,        % 左边距2.5厘米
    right=2.5cm,       % 右边距2.5厘米  
    top=2.5cm,         % 上边距2.5厘米
    bottom=2.5cm,      % 下边距2.5厘米
    headheight=1.5cm,  % 增加页眉高度以容纳校徽和黑线
    headsep=0.8cm,     % 页眉与正文间距
    footskip=1.2cm     % 页脚基准线偏移
}

% 页眉页脚和格式控制宏包
\usepackage{fancyhdr}        % 页眉页脚设置
\usepackage{lastpage}        % 获取总页数
\usepackage{titletoc}        % 目录格式控制
\usepackage{titlesec}        % 标题格式控制

% 字体设置
\usepackage{fontspec}  % 现代字体管理系统
\setmainfont{Times New Roman}  % 设置英文主字体为Times New Roman
% 中文字体设置
\setCJKmainfont{SimSun}[AutoFakeBold=3]  % 中文主字体:宋体
\setCJKsansfont{SimHei}[AutoFakeBold=3]  % 中文无衬线字体:黑体
\setCJKmonofont{KaiTi}[AutoFakeBold=3]   % 中文等宽字体:楷体

% 数学公式支持包
\usepackage{amsmath}    % 美国数学学会公式扩展
\usepackage{amssymb}    % 数学符号扩展包
\usepackage{amsthm}     % 定理环境支持

% 图形处理包
\usepackage{graphicx}   % 增强的图形支持
\graphicspath{{1_Frontmatter/}{2_Chapters/}{3_Backmatter/}}  % 设置图片搜索路径

% 表格处理包
\usepackage{booktabs}   % 出版质量的表格线
\usepackage{array}      % 增强的表格功能
\usepackage{longtable}  % 支持跨页长表格
\usepackage{multirow}   % 支持表格内多行合并

% 颜色支持包
\usepackage{xcolor}     % 颜色支持
\definecolor{linkblue}{RGB}{0,0,128}  % 定义链接蓝色

% 浮动体控制
\usepackage{float}      % 改进的浮动体定位
\usepackage{placeins}   % 提供浮动屏障控制

% 图表标题格式设置
\usepackage{caption}    % 自定义图表标题
\captionsetup{
    font=small,        % 字体大小
    labelfont=bf,      % 标签字体加粗
    textfont=normalfont, % 正文字体正常
    justification=centering, % 居中对齐
    labelsep=quad,     % 标签与标题间距
    skip=5pt          % 标题与图表间距
}

% 段落和缩进控制
\usepackage{indentfirst} % 确保段落首行缩进
\usepackage{setspace}   % 行距设置支持

% 参考文献处理包
\usepackage[numbers,sort&compress,super]{natbib}  % 添加super选项用于上标
\bibliographystyle{gbt7714-numerical}       % 中国国家标准参考文献样式
\setlength{\bibsep}{0.5ex}  % 设置参考文献行间距

% 其他实用工具包
\usepackage{enumitem}   % 自定义列表环境
\usepackage{verbatim}   % 代码环境支持
\usepackage{paralist}   % 紧凑列表环境

















% ========== 页眉页脚和页码格式设置 ==========
\pagestyle{fancy}  % 启用fancyhdr

% 定义页眉(所有页面都使用相同的页眉)
\newcommand{\setupheadermy}{%
    \fancyhf{} % 清除所有页眉页脚
    % 页眉设置:左边校徽,右边论文题目,下方黑线
    \fancyhead[L]{
        \raisebox{0cm}{\includegraphics[height=1cm]{1_Frontmatter/ShanghaiTech_Logo.png}}
    }
    \fancyhead[R]{
        \raisebox{0cm}{\zihao{-5}\heiti 本科毕业论文(设计)}
    }
    \renewcommand{\headrulewidth}{0.4pt}
    \renewcommand{\footrulewidth}{0pt}  % 页脚通常不需要横线
}

% 样式1:罗马数字页码(用于目录、摘要等前置部分)
\fancypagestyle{RomanStyle}{
    \setupheadermy
    \fancyfoot[C]{
        \zihao{5}\thepage
    }
}

% 样式2:阿拉伯数字页码(用于正文部分)
\fancypagestyle{ArabicStyle}{
    \setupheadermy
    \fancyfoot[C]{
        \zihao{5}第~\thepage~页~共~\pageref{LastPage}~页
    }
}

% 关键:重定义plain样式,使章节首页也有页眉
\fancypagestyle{plain}{
    \setupheadermy
    \fancyfoot[C]{}
}
\fancypagestyle{plain}{}  % 先清空











% ========== 章节标题格式设置 ==========
% 使用ctex宏包设置章节格式
\usepackage[zihao=5]{ctex}

% 设置章节标题格式
\ctexset{
    chapter = {
        name = {第,章},  % 设置章节名称为"第"和"章"
        number = \chinese{chapter},  % 使用阿拉伯数字编号
        nameformat = {},  % 清除默认格式
        format = \centering\zihao{3}\heiti\bfseries,  % 三号黑体居中加粗
        aftername = \quad,  % 编号和标题之间的间距
        beforeskip = 30pt,  % 章前间距(上空一行)
        afterskip = 30pt,  % 章后间距(下空一行)
        pagestyle = plain,  % 章节首页使用plain样式
    },
    section = {
        format = \zihao{4}\heiti\bfseries,  % 四号黑体
        indent = 0.75cm,  % 左缩进0.75厘米
        beforeskip = 3.5ex plus 1ex minus .2ex,
        afterskip = 2.3ex plus .2ex,
    },
    subsection = {
        format = \zihao{-4}\songti,  % 小四宋体
        indent = 2em,  % 缩进2字符
        beforeskip = 3.25ex plus 1ex minus .2ex,
        afterskip = 1.5ex plus .2ex,
    },
    subsubsection = {
        format = \zihao{-4}\songti,  % 小四宋体
        indent = 2em,  % 缩进2字符
        beforeskip = 3.25ex plus 1ex minus .2ex,
        afterskip = 1.5ex plus .2ex,
    },
    paragraph = {
        format = \zihao{5}\songti,
        indent = 2em,  % 首行缩进2字符
    }
}

% ========== 段落格式设置 ==========
\setlength{\parindent}{2em}    % 首行缩进2字符
\setlength{\parskip}{0pt}      % 段落间距为0
\linespread{1.5}               % 1.5倍行距
\renewcommand{\baselinestretch}{1.5}  % 确保1.5倍行距生效

% 设置中文和英文字体
\setCJKmainfont[AutoFakeBold=true]{SimSun}  % 中文主字体为宋体
\setCJKsansfont[AutoFakeBold=true]{SimHei}  % 中文无衬线字体为黑体
\setmainfont{Times New Roman}  % 英文主字体
\setsansfont{Arial}  % 英文无衬线字体

% 设置列表格式
\usepackage{enumitem}
\setlist[enumerate,1]{label=(\arabic*), leftmargin=2em}  % 一级列表格式 (1)
\setlist[enumerate,2]{label=\alph*), leftmargin=3em}    % 二级列表格式 a)
\setlist[enumerate,3]{label=\roman*), leftmargin=4em}   % 三级列表格式 i)
\setlist[itemize,1]{label=$\bullet$, leftmargin=2em}    % 一级项目符号
\setlist[itemize,2]{label=$\circ$, leftmargin=3em}      % 二级项目符号
\setlist{nosep}  % 去除列表间距

% 设置页边距
\usepackage{geometry}
\geometry{
    a4paper,
    left=3cm,
    right=2.5cm,
    top=3.5cm,
    bottom=2.5cm,
    headheight=1.5cm,
    headsep=1cm,
    footskip=1.5cm
}

% 设置图表标题格式
\usepackage{caption}
\DeclareCaptionFont{heiti}{\heiti}
\captionsetup[figure]{
    format=hang,
    font=small,
    labelsep=space,
    justification=centering
}
\captionsetup[table]{
    format=hang,
    font=small,
    labelsep=space,
    justification=centering
}

% 设置引用格式
\usepackage[numbers,sort&compress]{natbib}
\setcitestyle{super}  % 上标引用
\bibliographystyle{gbt7714-numerical}  % 使用国家标准GB/T 7714-2015




















% ========== 目录格式设置 ==========
% 定义必要的文档结构命令
\makeatletter
% 定义 frontmatter
\providecommand\frontmatter{%
  \cleardoublepage
  \@mainmatterfalse
  \pagenumbering{roman}  % 小写罗马数字
  \setcounter{page}{1}   % 从 i 开始
  \pagestyle{plain}     % 简单页码样式
}
% 定义 mainmatter
\providecommand\mainmatter{%
  \cleardoublepage
  \@mainmattertrue
  \pagenumbering{arabic}  % 阿拉伯数字页码
  \setcounter{page}{1}    % 从 1 开始
  \pagestyle{plain}      % 简单页码样式
}
% 定义 backmatter
\providecommand\backmatter{%
  \if@openright
    \cleardoublepage
  \else
    \clearpage
  \fi
  \@mainmatterfalse
}
\makeatother

% 重新定义目录标题格式
\makeatletter
% 保存原始的tableofcontents命令
\let\oldtableofcontents\tableofcontents
\renewcommand\tableofcontents{%
    \clearpage
    \pagestyle{RomanStyle}  % 添加这一行,确保目录页有页眉
    \phantomsection
    \addcontentsline{toc}{chapter}{\contentsname}
    \chapter*{%
        \vspace*{1\baselineskip}%
        \centering
        \zihao{3}\heiti\bfseries
        目\quad 录%
        \vspace*{1\baselineskip}%
    }%
    \markboth{目录}{目录}%
    \@starttoc{toc}%
}
\makeatother

% 设置目录格式
\RequirePackage{titletoc}
% 章标题格式(一级标题)
\titlecontents{chapter}
    [0pt]                                    % 左侧缩进
    {\addvspace{6pt}\zihao{5}\songti\bfseries} % 前间距和字体
    {\thecontentslabel\hspace{0.5em}}       % 标签宽度
    {}                                       % 无编号格式
    {\titlerule*[0.5em]{.}\contentspage}    % 引导点和页码

% 节标题格式(二级标题)
\titlecontents{section}
    [2em]                                    % 左侧缩进
    {\addvspace{2pt}\zihao{5}\songti}       % 前间距和字体
    {\thecontentslabel\hspace{0.5em}}       % 标签宽度
    {}                                       % 无编号格式
    {\titlerule*[0.5em]{.}\contentspage}    % 引导点和页码

% 小节标题格式(三级标题)
\titlecontents{subsection}
    [4.5em]                                  % 左侧缩进
    {\addvspace{1pt}\zihao{5}\songti}       % 前间距和字体
    {\thecontentslabel\hspace{0.5em}}       % 标签宽度
    {}                                       % 无编号格式
    {\titlerule*[0.5em]{.}\contentspage}    % 引导点和页码






% ========== 超链接样式设置(必须最后加载) ==========
\usepackage[]{hyperref}% 创建超链接和书签
\usepackage{bookmark}                % 增强的书签支持

\hypersetup{
    colorlinks=true,        % 使用彩色链接
    linkcolor=black,        % 内部链接黑色
    citecolor=black,        % 引用链接黑色  
    filecolor=black,        % 文件链接黑色
    urlcolor=blue,          % URL链接蓝色
    bookmarksnumbered=true, % 书签显示编号
    bookmarksopen=true      % 展开书签
}


% ========== 其他格式设置 ==========

% 表格线粗细设置
\setlength{\heavyrulewidth}{0.08em}   % 粗线宽度
\setlength{\lightrulewidth}{0.05em}   % 细线宽度  
\setlength{\cmidrulewidth}{0.03em}    % 中间线宽度

% 防止孤行和寡行设置
\clubpenalty=10000    % 防止段首孤行
\widowpenalty=10000   % 防止段末寡行
\brokenpenalty=10000  % 防止断字

% 图表标题中文设置
\renewcommand{\figurename}{图}     % 图标题改为中文"图"
\renewcommand{\tablename}{表}      % 表标题改为中文"表"
\renewcommand{\contentsname}{目录}  % 目录标题改为中文"目录"
\renewcommand{\abstractname}{摘要}  % 摘要标题改为中文"摘要"
\renewcommand{\bibname}{参考文献}  % 参考文献标题

% 数学环境允许跨页
\allowdisplaybreaks[4]  % 允许公式跨页显示,优先级为4

% 优化浮动体放置参数
\renewcommand{\floatpagefraction}{0.8}  % 浮动页中浮动体最小占比
\renewcommand{\textfraction}{0.1}        % 文本页中文本最小占比
\renewcommand{\topfraction}{0.9}         % 页顶可放置浮动体的最大比例

% 自定义命令:用于在main.tex中切换页码样式
\newcommand{\setarabicpagination}{
    \pagestyle{shanghaitech}
    \fancyfoot[C]{\zihao{5}第~\thepage~页~共~\pageref{LastPage}~页}
    \pagenumbering{arabic}
}

\newcommand{\setromanpagination}{
    \pagestyle{shanghaitech}
    \fancyfoot[C]{\zihao{5}\thepage}
    \pagenumbering{Roman}
}