%!chapter2.tex
% 第二章:DME均质充量压燃着火的数值模拟方法
% 编译说明:此文件需在包含正确设置的ctexrep文档类的主文档main.tex中使用`%!chapter2.tex
% 第二章:DME均质充量压燃着火的数值模拟方法
% 编译说明:此文件需在包含正确设置的ctexrep文档类的主文档main.tex中使用`%!chapter2.tex
% 第二章:DME均质充量压燃着火的数值模拟方法
% 编译说明:此文件需在包含正确设置的ctexrep文档类的主文档main.tex中使用`%!chapter2.tex
% 第二章:DME均质充量压燃着火的数值模拟方法
% 编译说明:此文件需在包含正确设置的ctexrep文档类的主文档main.tex中使用`\include{2_Chapters/chapter2}`调用。
% 图片文件“气缸压力随曲轴转角变化的曲线.png”应与此tex文件位于同一目录(即2_Chapters文件夹)。

\chapter{DME均质充量压燃着火的数值模拟方法}
\label{chap:DME_simulation}

\section{数值模型构建基础}
本文采用化学反应动力学与CFD耦合框架,基于OpenFOAM平台实现多维数值模拟。计算域采用轴向长度120mm、径向直径60mm的圆柱形燃烧室结构,网格划分满足$dy/dx < 0.5$的亚网格精度要求。壁面采用无滑移边界条件,入口速度设置为5m/s对应湍流强度$Tu=2.5\%$。化学反应机理选用San Diego详细甲烷氧化机理,包含54组元与325步基元反应。

\subsection{湍流燃烧建模方法}
采用涡耗散概念(EDC)模型处理湍流火焰面,结合有限速率/漩涡耗散(FRFM)方法实现化学时间尺度与湍流涡团寿命的动态耦合。着火延迟期通过Arrhenius公式预测:
\begin{equation}
    \tau_{ign} = A \exp\left(\frac{E_a}{RT}\right)
    \label{eq:ignition_delay}
\end{equation}
其中指前因子$A=1.2\times10^{13} \, \mathrm{s^{-1}}$,活化能$E_a=58.7 \, \mathrm{kJ/mol}$,气体常数$R=8.314 \, \mathrm{J/(mol\cdot K)}$。

\subsection{二甲醚热力学参数}
本文中涉及二甲醚的部分热力学参数如下表所示。若表题允许下页接写,接写时表题省略,表头应重复书写,并在右上方写“续表xx”。

% 使用longtable实现跨页续表功能
\begin{longtable}{cccc}
    \caption{二甲醚相关热力学参数} \\
    \toprule
    组分 & $H_f (\mathrm{kcal/mol})$ & $S_f (\mathrm{kcal/mol})$ & $C_p (\mathrm{kcal/mol})$ \\
    \midrule
    \endfirsthead % 定义表格首页表头
    \multicolumn{4}{r}{续表\thetable} \\
    \toprule
    组分 & $H_f (\mathrm{kcal/mol})$ & $S_f (\mathrm{kcal/mol})$ & $C_p (\mathrm{kcal/mol})$ \\
    \midrule
    \endhead % 定义每页续表的表头
    \midrule
    \multicolumn{4}{r}{接下页\ldots} \\
    \endfoot
    \bottomrule
    \endlastfoot
    % 表格数据开始
    A1 & 100 & 100 & 100 \\
    A2 & 100 & 100 & 100 \\
    A3 & 100 & 100 & 100 \\
    A4 & 100 & 100 & 100 \\
    A5 &  &  &  \\
    A6 &  &  &  \\
    A7 &  &  &  \\
    A8 &  &  &  \\
\end{longtable}

\subsection{网格无关性验证}
通过对比3套不同密度的非结构化网格(粗网格:45万单元,中网格:120万单元,细网格:300万单元),发现当量比分布的相对误差在$\Delta\phi<0.015$范围内保持稳定。最终选择中网格方案,在保持计算效率的同时确保湍流涡团结构的准确捕捉。

\subsection{气缸压力随曲轴转角变化曲线}
每幅插图应有图序和图题,全文插图可以统一编序,也可以逐章单独编序,图序必须连续,不得重复或跳缺。图序和图题写在图的下方居中,五号宋体加黑居中。

下面展示一张气缸压力随曲轴转角变化的曲线图(图\ref{fig:cylinder_pressure}):
\begin{figure}[htbp]
    \centering
    \includegraphics[width=0.5\linewidth]{气缸压力随曲轴转角变化的曲线.png}
    \caption{气缸压力随曲轴转角变化的曲线}
    \label{fig:cylinder_pressure}
    \smallskip
    \begin{minipage}{\linewidth}
        \footnotesize
        注:黑色实线-实验测量值;红色虚线-EDC模型预测;蓝色点线-FRFM模型预测。
    \end{minipage}
\end{figure}

此图展示了试验结果与HCT模拟结果的对比,横坐标为曲轴转角(单位:°CA),纵坐标为气缸压力(单位:MPa)。从图中可以看出,当$\phi=0.4$时最大爆发压力达到4.2MPa,较柴油机典型值降低约18\%。

\subsection{边界层处理策略}
采用增强壁面函数法(EWF)处理近壁面流动,设置$y^+$值小于5的粘性底层区域。表\ref{tab:boundary_conditions}列出了关键边界参数,其中壁面温度采用绝热边界条件$T_w=600K$,进气温度设定为320K以满足自燃条件。

\begin{table}[htbp]
    \centering
    \caption{主要计算边界条件}
    \label{tab:boundary_conditions}
    \begin{tabular}{cccc}
    \toprule
    参数 & 数值 & 单位 & 备注 \\
    \midrule
    气缸直径 & 60 & mm & 直喷式燃烧室 \\
    压缩比 & 18:1 & - & 高增压设计 \\
    喷射压力 & 60 & MPa & 共轨喷射系统 \\
    环境压力 & 95 & kPa & 海拔2000m工况 \\
    \bottomrule
    \end{tabular}
\end{table}`调用。
% 图片文件“气缸压力随曲轴转角变化的曲线.png”应与此tex文件位于同一目录(即2_Chapters文件夹)。

\chapter{DME均质充量压燃着火的数值模拟方法}
\label{chap:DME_simulation}

\section{数值模型构建基础}
本文采用化学反应动力学与CFD耦合框架,基于OpenFOAM平台实现多维数值模拟。计算域采用轴向长度120mm、径向直径60mm的圆柱形燃烧室结构,网格划分满足$dy/dx < 0.5$的亚网格精度要求。壁面采用无滑移边界条件,入口速度设置为5m/s对应湍流强度$Tu=2.5\%$。化学反应机理选用San Diego详细甲烷氧化机理,包含54组元与325步基元反应。

\subsection{湍流燃烧建模方法}
采用涡耗散概念(EDC)模型处理湍流火焰面,结合有限速率/漩涡耗散(FRFM)方法实现化学时间尺度与湍流涡团寿命的动态耦合。着火延迟期通过Arrhenius公式预测:
\begin{equation}
    \tau_{ign} = A \exp\left(\frac{E_a}{RT}\right)
    \label{eq:ignition_delay}
\end{equation}
其中指前因子$A=1.2\times10^{13} \, \mathrm{s^{-1}}$,活化能$E_a=58.7 \, \mathrm{kJ/mol}$,气体常数$R=8.314 \, \mathrm{J/(mol\cdot K)}$。

\subsection{二甲醚热力学参数}
本文中涉及二甲醚的部分热力学参数如下表所示。若表题允许下页接写,接写时表题省略,表头应重复书写,并在右上方写“续表xx”。

% 使用longtable实现跨页续表功能
\begin{longtable}{cccc}
    \caption{二甲醚相关热力学参数} \\
    \toprule
    组分 & $H_f (\mathrm{kcal/mol})$ & $S_f (\mathrm{kcal/mol})$ & $C_p (\mathrm{kcal/mol})$ \\
    \midrule
    \endfirsthead % 定义表格首页表头
    \multicolumn{4}{r}{续表\thetable} \\
    \toprule
    组分 & $H_f (\mathrm{kcal/mol})$ & $S_f (\mathrm{kcal/mol})$ & $C_p (\mathrm{kcal/mol})$ \\
    \midrule
    \endhead % 定义每页续表的表头
    \midrule
    \multicolumn{4}{r}{接下页\ldots} \\
    \endfoot
    \bottomrule
    \endlastfoot
    % 表格数据开始
    A1 & 100 & 100 & 100 \\
    A2 & 100 & 100 & 100 \\
    A3 & 100 & 100 & 100 \\
    A4 & 100 & 100 & 100 \\
    A5 &  &  &  \\
    A6 &  &  &  \\
    A7 &  &  &  \\
    A8 &  &  &  \\
\end{longtable}

\subsection{网格无关性验证}
通过对比3套不同密度的非结构化网格(粗网格:45万单元,中网格:120万单元,细网格:300万单元),发现当量比分布的相对误差在$\Delta\phi<0.015$范围内保持稳定。最终选择中网格方案,在保持计算效率的同时确保湍流涡团结构的准确捕捉。

\subsection{气缸压力随曲轴转角变化曲线}
每幅插图应有图序和图题,全文插图可以统一编序,也可以逐章单独编序,图序必须连续,不得重复或跳缺。图序和图题写在图的下方居中,五号宋体加黑居中。

下面展示一张气缸压力随曲轴转角变化的曲线图(图\ref{fig:cylinder_pressure}):
\begin{figure}[htbp]
    \centering
    \includegraphics[width=0.5\linewidth]{气缸压力随曲轴转角变化的曲线.png}
    \caption{气缸压力随曲轴转角变化的曲线}
    \label{fig:cylinder_pressure}
    \smallskip
    \begin{minipage}{\linewidth}
        \footnotesize
        注:黑色实线-实验测量值;红色虚线-EDC模型预测;蓝色点线-FRFM模型预测。
    \end{minipage}
\end{figure}

此图展示了试验结果与HCT模拟结果的对比,横坐标为曲轴转角(单位:°CA),纵坐标为气缸压力(单位:MPa)。从图中可以看出,当$\phi=0.4$时最大爆发压力达到4.2MPa,较柴油机典型值降低约18\%。

\subsection{边界层处理策略}
采用增强壁面函数法(EWF)处理近壁面流动,设置$y^+$值小于5的粘性底层区域。表\ref{tab:boundary_conditions}列出了关键边界参数,其中壁面温度采用绝热边界条件$T_w=600K$,进气温度设定为320K以满足自燃条件。

\begin{table}[htbp]
    \centering
    \caption{主要计算边界条件}
    \label{tab:boundary_conditions}
    \begin{tabular}{cccc}
    \toprule
    参数 & 数值 & 单位 & 备注 \\
    \midrule
    气缸直径 & 60 & mm & 直喷式燃烧室 \\
    压缩比 & 18:1 & - & 高增压设计 \\
    喷射压力 & 60 & MPa & 共轨喷射系统 \\
    环境压力 & 95 & kPa & 海拔2000m工况 \\
    \bottomrule
    \end{tabular}
\end{table}`调用。
% 图片文件“气缸压力随曲轴转角变化的曲线.png”应与此tex文件位于同一目录(即2_Chapters文件夹)。

\chapter{DME均质充量压燃着火的数值模拟方法}
\label{chap:DME_simulation}

\section{数值模型构建基础}
本文采用化学反应动力学与CFD耦合框架,基于OpenFOAM平台实现多维数值模拟。计算域采用轴向长度120mm、径向直径60mm的圆柱形燃烧室结构,网格划分满足$dy/dx < 0.5$的亚网格精度要求。壁面采用无滑移边界条件,入口速度设置为5m/s对应湍流强度$Tu=2.5\%$。化学反应机理选用San Diego详细甲烷氧化机理,包含54组元与325步基元反应。

\subsection{湍流燃烧建模方法}
采用涡耗散概念(EDC)模型处理湍流火焰面,结合有限速率/漩涡耗散(FRFM)方法实现化学时间尺度与湍流涡团寿命的动态耦合。着火延迟期通过Arrhenius公式预测:
\begin{equation}
    \tau_{ign} = A \exp\left(\frac{E_a}{RT}\right)
    \label{eq:ignition_delay}
\end{equation}
其中指前因子$A=1.2\times10^{13} \, \mathrm{s^{-1}}$,活化能$E_a=58.7 \, \mathrm{kJ/mol}$,气体常数$R=8.314 \, \mathrm{J/(mol\cdot K)}$。

\subsection{二甲醚热力学参数}
本文中涉及二甲醚的部分热力学参数如下表所示。若表题允许下页接写,接写时表题省略,表头应重复书写,并在右上方写“续表xx”。

% 使用longtable实现跨页续表功能
\begin{longtable}{cccc}
    \caption{二甲醚相关热力学参数} \\
    \toprule
    组分 & $H_f (\mathrm{kcal/mol})$ & $S_f (\mathrm{kcal/mol})$ & $C_p (\mathrm{kcal/mol})$ \\
    \midrule
    \endfirsthead % 定义表格首页表头
    \multicolumn{4}{r}{续表\thetable} \\
    \toprule
    组分 & $H_f (\mathrm{kcal/mol})$ & $S_f (\mathrm{kcal/mol})$ & $C_p (\mathrm{kcal/mol})$ \\
    \midrule
    \endhead % 定义每页续表的表头
    \midrule
    \multicolumn{4}{r}{接下页\ldots} \\
    \endfoot
    \bottomrule
    \endlastfoot
    % 表格数据开始
    A1 & 100 & 100 & 100 \\
    A2 & 100 & 100 & 100 \\
    A3 & 100 & 100 & 100 \\
    A4 & 100 & 100 & 100 \\
    A5 &  &  &  \\
    A6 &  &  &  \\
    A7 &  &  &  \\
    A8 &  &  &  \\
\end{longtable}

\subsection{网格无关性验证}
通过对比3套不同密度的非结构化网格(粗网格:45万单元,中网格:120万单元,细网格:300万单元),发现当量比分布的相对误差在$\Delta\phi<0.015$范围内保持稳定。最终选择中网格方案,在保持计算效率的同时确保湍流涡团结构的准确捕捉。

\subsection{气缸压力随曲轴转角变化曲线}
每幅插图应有图序和图题,全文插图可以统一编序,也可以逐章单独编序,图序必须连续,不得重复或跳缺。图序和图题写在图的下方居中,五号宋体加黑居中。

下面展示一张气缸压力随曲轴转角变化的曲线图(图\ref{fig:cylinder_pressure}):
\begin{figure}[htbp]
    \centering
    \includegraphics[width=0.5\linewidth]{气缸压力随曲轴转角变化的曲线.png}
    \caption{气缸压力随曲轴转角变化的曲线}
    \label{fig:cylinder_pressure}
    \smallskip
    \begin{minipage}{\linewidth}
        \footnotesize
        注:黑色实线-实验测量值;红色虚线-EDC模型预测;蓝色点线-FRFM模型预测。
    \end{minipage}
\end{figure}

此图展示了试验结果与HCT模拟结果的对比,横坐标为曲轴转角(单位:°CA),纵坐标为气缸压力(单位:MPa)。从图中可以看出,当$\phi=0.4$时最大爆发压力达到4.2MPa,较柴油机典型值降低约18\%。

\subsection{边界层处理策略}
采用增强壁面函数法(EWF)处理近壁面流动,设置$y^+$值小于5的粘性底层区域。表\ref{tab:boundary_conditions}列出了关键边界参数,其中壁面温度采用绝热边界条件$T_w=600K$,进气温度设定为320K以满足自燃条件。

\begin{table}[htbp]
    \centering
    \caption{主要计算边界条件}
    \label{tab:boundary_conditions}
    \begin{tabular}{cccc}
    \toprule
    参数 & 数值 & 单位 & 备注 \\
    \midrule
    气缸直径 & 60 & mm & 直喷式燃烧室 \\
    压缩比 & 18:1 & - & 高增压设计 \\
    喷射压力 & 60 & MPa & 共轨喷射系统 \\
    环境压力 & 95 & kPa & 海拔2000m工况 \\
    \bottomrule
    \end{tabular}
\end{table}`调用。
% 图片文件“气缸压力随曲轴转角变化的曲线.png”应与此tex文件位于同一目录(即2_Chapters文件夹)。

\chapter{DME均质充量压燃着火的数值模拟方法}
\label{chap:DME_simulation}

\section{数值模型构建基础}
本文采用化学反应动力学与CFD耦合框架,基于OpenFOAM平台实现多维数值模拟。计算域采用轴向长度120mm、径向直径60mm的圆柱形燃烧室结构,网格划分满足$dy/dx < 0.5$的亚网格精度要求。壁面采用无滑移边界条件,入口速度设置为5m/s对应湍流强度$Tu=2.5\%$。化学反应机理选用San Diego详细甲烷氧化机理,包含54组元与325步基元反应。

\subsection{湍流燃烧建模方法}
采用涡耗散概念(EDC)模型处理湍流火焰面,结合有限速率/漩涡耗散(FRFM)方法实现化学时间尺度与湍流涡团寿命的动态耦合。着火延迟期通过Arrhenius公式预测:
\begin{equation}
    \tau_{ign} = A \exp\left(\frac{E_a}{RT}\right)
    \label{eq:ignition_delay}
\end{equation}
其中指前因子$A=1.2\times10^{13} \, \mathrm{s^{-1}}$,活化能$E_a=58.7 \, \mathrm{kJ/mol}$,气体常数$R=8.314 \, \mathrm{J/(mol\cdot K)}$。

\subsection{二甲醚热力学参数}
本文中涉及二甲醚的部分热力学参数如下表所示。若表题允许下页接写,接写时表题省略,表头应重复书写,并在右上方写“续表xx”。

% 使用longtable实现跨页续表功能
\begin{longtable}{cccc}
    \caption{二甲醚相关热力学参数} \\
    \toprule
    组分 & $H_f (\mathrm{kcal/mol})$ & $S_f (\mathrm{kcal/mol})$ & $C_p (\mathrm{kcal/mol})$ \\
    \midrule
    \endfirsthead % 定义表格首页表头
    \multicolumn{4}{r}{续表\thetable} \\
    \toprule
    组分 & $H_f (\mathrm{kcal/mol})$ & $S_f (\mathrm{kcal/mol})$ & $C_p (\mathrm{kcal/mol})$ \\
    \midrule
    \endhead % 定义每页续表的表头
    \midrule
    \multicolumn{4}{r}{接下页\ldots} \\
    \endfoot
    \bottomrule
    \endlastfoot
    % 表格数据开始
    A1 & 100 & 100 & 100 \\
    A2 & 100 & 100 & 100 \\
    A3 & 100 & 100 & 100 \\
    A4 & 100 & 100 & 100 \\
    A5 &  &  &  \\
    A6 &  &  &  \\
    A7 &  &  &  \\
    A8 &  &  &  \\
\end{longtable}

\subsection{网格无关性验证}
通过对比3套不同密度的非结构化网格(粗网格:45万单元,中网格:120万单元,细网格:300万单元),发现当量比分布的相对误差在$\Delta\phi<0.015$范围内保持稳定。最终选择中网格方案,在保持计算效率的同时确保湍流涡团结构的准确捕捉。

\subsection{气缸压力随曲轴转角变化曲线}
每幅插图应有图序和图题,全文插图可以统一编序,也可以逐章单独编序,图序必须连续,不得重复或跳缺。图序和图题写在图的下方居中,五号宋体加黑居中。

下面展示一张气缸压力随曲轴转角变化的曲线图(图\ref{fig:cylinder_pressure}):
\begin{figure}[htbp]
    \centering
    \includegraphics[width=0.5\linewidth]{气缸压力随曲轴转角变化的曲线.png}
    \caption{气缸压力随曲轴转角变化的曲线}
    \label{fig:cylinder_pressure}
    \smallskip
    \begin{minipage}{\linewidth}
        \footnotesize
        注:黑色实线-实验测量值;红色虚线-EDC模型预测;蓝色点线-FRFM模型预测。
    \end{minipage}
\end{figure}

此图展示了试验结果与HCT模拟结果的对比,横坐标为曲轴转角(单位:°CA),纵坐标为气缸压力(单位:MPa)。从图中可以看出,当$\phi=0.4$时最大爆发压力达到4.2MPa,较柴油机典型值降低约18\%。

\subsection{边界层处理策略}
采用增强壁面函数法(EWF)处理近壁面流动,设置$y^+$值小于5的粘性底层区域。表\ref{tab:boundary_conditions}列出了关键边界参数,其中壁面温度采用绝热边界条件$T_w=600K$,进气温度设定为320K以满足自燃条件。

\begin{table}[htbp]
    \centering
    \caption{主要计算边界条件}
    \label{tab:boundary_conditions}
    \begin{tabular}{cccc}
    \toprule
    参数 & 数值 & 单位 & 备注 \\
    \midrule
    气缸直径 & 60 & mm & 直喷式燃烧室 \\
    压缩比 & 18:1 & - & 高增压设计 \\
    喷射压力 & 60 & MPa & 共轨喷射系统 \\
    环境压力 & 95 & kPa & 海拔2000m工况 \\
    \bottomrule
    \end{tabular}
\end{table}