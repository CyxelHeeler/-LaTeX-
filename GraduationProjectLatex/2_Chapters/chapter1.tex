% 第一章
\chapter{绪论}

随着汽车工业的发展和汽车保有量的增加,
汽车在大量消耗石油燃料的同时,
尾气排出的有害气体还严重地污染了人们赖以生存的大气环境,
实现能源与环境长期可持续发展是摆在汽车和内燃机工作者面前的重大课题。
环保和能源是发动机工业需要解决的两个主要问题。
目前,随着人们对环境污染重视程度的日益提高,
各国越来越重视环境保护,
现在已制定了将NOx和PM视为大气污染源的强化法规,
如美国加州在1998年生效的一项超低排放汽车法规规定汽车的NOx+HC
排放<2.5g/bph-hr, PM排放<0.05g/bph-hr。为满足严格的排放要求,
研究人员在各个相关领域进行了大量的研究工作,
改进发动机的燃烧系统作为一个重要解决途径,也取得了一定进展\cite{chen2000}。

传统汽油机均质混合气,尾气排放污染物主要包括氮氧化物(NOx)、碳氢化合物(HC)、一氧化碳(CO),可以通过三效催化后处理加以解决,但要达到欧IV及其以上标准仍存在较大困难,且汽油机的热效率低,在中低负荷工作时还有较大的泵气损失。柴油机热效率高,但排气中的NOx和碳烟微粒排放物(PM)却难以折中,使用一种排放物减少的措施,往往导致另一排放物的增加。由于柴油机总体上富氧燃烧, NOx的催化处理技术尚未成熟。汽油机和柴油机的燃烧方式都不能解决碳烟和氮氧化物生成的trade-off关系,因而很难在这两种燃烧模式下通过改进燃烧来同时大量降低碳烟和氮氧化物的生成。

\section{HCCI的数值模拟研究现状}

HCCI发动机的着火与燃烧过程与传统的火花塞点火式和压燃式发动机有着本质的区别,在HCCI发动机的着火燃烧过程中,燃料的化学反应动力学起着至关重要的作用。因此,相对于传统发动机数值模拟研究主要侧重于湍流混合与燃烧模型而言,HCCI发动机燃烧模拟的焦点主要集中在燃料的反应机理和化学动力学模型上。

\subsection{HCCI数值模拟模型}

目前HCCI数值模拟研究主要集中在单区、多区和多维模型上\cite{li2000}。
本节将从这三方面分别予以介绍:

\noindent
(1) 单区模型

\indent
单区模型是最简单的HCCI燃烧模型,它假设燃烧室内各点具有相同的热力学状态,不考虑温度、压力和组分浓度的空间分布。该模型将整个燃烧室视为一个均匀的热力学系统,通过求解质量、能量和组分守恒方程来预测燃烧过程。单区模型计算效率高,适用于参数化研究和燃烧特性分析,但由于忽略了燃烧室内温度、浓度和流动的不均匀性,无法准确预测点火时刻、燃烧持续期和排放生成。

…………………………………………………………………

\noindent
(2) 双区和多区模型

\indent
双区和多区模型在单区模型基础上发展而来,旨在考虑燃烧室内温度、浓度和反应速率的非均匀性。双区模型将燃烧室划分为未燃区和已燃区,多区模型则进一步将燃烧室划分为多个区域。这些模型通过考虑不同区域之间的质量、能量和组分交换,能够更准确地描述HCCI燃烧过程中的温度梯度和反应速率差异,从而提高燃烧参数预测精度。

…………………………………………………………………

\noindent
(3) 多维模型

\indent
多维模型(如CFD模型)是HCCI燃烧模拟中最详细的模型,它通过求解三维Navier-Stokes方程、能量方程、组分输运方程和湍流模型,能够完整描述燃烧室内流动、传热、传质和化学反应过程。多维模型可以考虑燃烧室几何形状、进气流动、湍流混合、壁面传热等复杂因素,提供最全面的燃烧信息,但计算成本极高,通常用于机理研究和特定工况分析。

…………………………………………………………………